\documentclass[spanish]{beamer}

%%% CODIFICACIÓN

%\usepackage[x11names, rgb, html]{xcolor}
\usepackage[utf8]{inputenc}
\usepackage[spanish]{babel}
\usepackage{graphics}

%%% FUENTES

\usepackage[T1]{fontenc}
%\usepackage[familydefault,regular]{Chivo}
%\usepackage{newtxsf} % Fuente de matemáticas
%\usepackage{mathastext}
%\usepackage[scaled=.85]{FiraMono}

\setbeamertemplate{navigation symbols}{}

%%% COLORES

\definecolor{50}{HTML}{FFEBEE}
\definecolor{100}{HTML}{FFCDD2}
\definecolor{200}{HTML}{EF9A9A}
\definecolor{300}{HTML}{E57373}
\definecolor{400}{HTML}{EF5350}
\definecolor{500}{HTML}{F44336}
\definecolor{600}{HTML}{E53935}
\definecolor{700}{HTML}{D32F2F}
\definecolor{800}{HTML}{C62828}
\definecolor{900}{HTML}{B71C1C}

%% Colores de Solarized

\definecolor{sbase03}{HTML}{002B36}
\definecolor{sbase02}{HTML}{073642}
\definecolor{sbase01}{HTML}{586E75}
\definecolor{sbase00}{HTML}{657B83}
\definecolor{sbase0}{HTML}{839496}
\definecolor{sbase1}{HTML}{93A1A1}
\definecolor{sbase2}{HTML}{EEE8D5}
\definecolor{sbase3}{HTML}{FDF6E3}
\definecolor{syellow}{HTML}{B58900}
\definecolor{sorange}{HTML}{CB4B16}
\definecolor{sred}{HTML}{DC322F}
\definecolor{smagenta}{HTML}{D33682}
\definecolor{sviolet}{HTML}{6C71C4}
\definecolor{sblue}{HTML}{268BD2}
\definecolor{scyan}{HTML}{2AA198}
\definecolor{sgreen}{HTML}{859900}

%% Colores del documento

\definecolor{background}{RGB}{237,237,237}
\definecolor{text}{RGB}{78,78,78}
\definecolor{accent}{RGB}{129, 26, 24}
\definecolor{accent2}{HTML}{814918}
\definecolor{accent3}{HTML}{136618}
\definecolor{accent4}{HTML}{0F4B4E}
\definecolor{accent5}{HTML}{681341}
\definecolor{accent6}{HTML}{1F1B5A}

\setbeamerfont{framesubtitle}{size=\normalfont\small}
\setbeamercolor{framesubtitle}{fg=white}

%%% AJUSTES DE BEAMER

% ¿Negrita en el título de diapositiva o no?
%\setbeamertemplate{frametitle}{\color{accent}\vspace*{1cm}\bfseries\insertframetitle\par\vskip-6pt}

\setbeamertemplate{frametitle}{\color{900}\vspace*{1cm}\insertframetitle\\\usebeamerfont{framesubtitle}\insertframesubtitle\par\vskip-6pt}

\setbeamertemplate{itemize items}[circle] % Viñetas de itemize

%%% CONFIGURACIÓN DE COLORES DE BEAMER

\setbeamercolor{background canvas}{bg=background}
\setbeamercolor{normal text}{fg=text}
\setbeamercolor{alerted text}{fg=900}
\setbeamercolor{block title}{fg=900}
\setbeamercolor{alerted text}{fg=900}
\setbeamercolor{itemize item}{fg=900}
\setbeamercolor{enumerate item}{fg=900}
\setbeamercolor*{title}{fg=900}
\setbeamercolor{qed symbol}{fg=900}
\usebeamercolor[fg]{normal text}

%%% INFORMACIÓN DEL DOCUMENTO

\title{Fundamentos de Redes}
\subtitle{Redes anónimas: I2P}
\author{Alberto Jesús Durán López\\ Antonio Coín Castro\\ \vspace{1em}Grupo 5}
\begin{document}


\maketitle

%%% Inicio diapositiva
\begin{frame}{Anonimato en el acceso corriente a internet}%{Subtítulo (opcional)}

Hableremos sobre distintas formas de obtener anonimato: \\ %Puedes comentar esta línea si quieres

\vspace{1.5em} %% espaciado vertical

\begin{itemize}
	\item Proxy, VPN...
    \item I2P, Invisible Internet Project \\ 
\end{itemize}

\vspace{1.5em} %% espaciado vertical

%El espaciado vertical se controla con "$\textbackslash$vspace$\{1.2em\}$", modificando el número de dentro.

\end{frame}  
%%% Fin diapositiva





\begin{frame}
	Podemos poner una diapos. con una foto introductoria para el simil de cuando
	nos conectamos a la red - plaza , if u want it would be great
\end{frame}







\begin{frame}{Métodos para mantener el anonimato}
	
\begin{itemize}
	\item Proxy \\
	\item VPN - Virtual Private Network \\ 
	\item Otros
\end{itemize}	
	
	
\end{frame}






\begin{frame}{Proxy}
	
Servidor intermediario entre las conexiones de un cliente y un servidor. \\

\vspace{2.9em}
\begin{itemize}
	\item Dirección IP camuflada \\
	\item Acceso a contenido bloqueados de algunos países \\ 
\end{itemize}
	
\end{frame}





\begin{frame}{VPN}

Medio de extender una red privada a través de una red pública

\vspace{1.9em}

\begin{itemize}
	\item Se puede acceder a una red privada remotamente \\
	\item Dirección IP camuflada \\ 
	\item Confidencialidad garantizada - Paquetes encriptados \\
	\item Sistema de autentificación para conectarse
	\item Mecanismos para mantener integridad de mensajes
\end{itemize}	

\end{frame}






\begin{frame}{Otros...}

\begin{itemize}
	\item Navegadores privados: DuckDuckGo
	\item Buscar otro ej. NO tengo internet here D:
\end{itemize}

\end{frame}



\begin{frame}{I2P, Invisible Internet Project}
	
Se puede poner diapositiva introductoria a I2P con foto	
	
\end{frame}




\begin{frame}{I2P, Invisible Internet Project}
	
	
\begin{itemize}
	\item Nodos y túneles
	\item Enrutamiento tipo garlic
	\item Encriptación de la información
\end{itemize}	
	
\end{frame}






\begin{frame}{I2P, Invisible Internet Project}
	
	Nodos y túneles for you
	and kademlia too	
	
\end{frame}






\begin{frame}{I2P, Invisible Internet Project}
	
Enrutamiento tipo garlic:


\vspace{1.9em}


\begin{itemize}
	\item Cifrado por capas
	\item Agregación de múltiples mensajes juntos
	\item Cifrado ElGamal/AES
\end{itemize}
	
	
\end{frame}



\begin{frame}{I2P, Invisible Internet Project}

gonna study topología, see ya I2P!

\end{frame}




\end{document}
